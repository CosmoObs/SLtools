\section{Introduction}

General structure will be:

--- Intro

    - motivation

    - objecives

    - outline

--- SLPM
   
   - Brief Overview of method
     
       - main fields
 
       - lens curves

       - image reconstruction

       - meast of image properties

       - axisymmetric and PE models


--- Comparison with fully numerical soln

      - Comparison of gof and H-dorff measures on lenscurves, to eps max cutoff

      - Comparison of image properties (L,W,A)
         
      - Application to lenscurves + arcs for 4 basic models (sis, psis, nfw, pnfw)



- Conclusion and projections for the future


-------------------------------------------- 

[Based on Bruno suggestion on the beginning:]


In a previous paper [Alard], a perturbative method was introduced with
the aim of obtaining analytic approximations to some Strong Lensing
models. We apply this method to axisymmetric and pseudo-elliptical
isothermal and Navarro-Frenk-White models. In order to compare these
results with full numerical solutions, we define two new distance
measurements for the difference between CCs and caustics. We also
define several geometrical arc properties. We use all these
definitions to define limits of validity to the perturbative method
and we show that they are consistent with older limits introduced in
[Alard C3].
