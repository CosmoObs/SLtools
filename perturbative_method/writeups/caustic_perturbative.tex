\documentclass[a4paper,twoside]{article}
\usepackage{latexsym}
\usepackage[pdftex]{color,graphicx}
\usepackage{latexsym}
\usepackage{subfigure}
\usepackage{amsmath}
\usepackage{ccaption}
\usepackage{natbib}
\usepackage{amssymb}
\usepackage[brazilian]{babel}
\usepackage[latin1]{inputenc}
\def \re {r_{\mathrm{E}}}
\def \al {\alpha}
\def \te {\theta}
\def \tep {\theta_p}
\def \kt {\kappa_2}
\def \mf {\mathcal{F}}
\def \mg {\mathcal{G}}
\def \pa {\partial}
\begin{document}
\title{Caustic in the perturbative approach \\ Verification of the Sect. 4 from Alard�s Paper 2007}
%
\author{Habib S. Dumet-Montoya}
% email{hdumetm@cbpf.br}
\affiliation{ICRA-CBPF}
% \author{Mart\'in Makler}
% \email{martin@cbpf.br}
% \affiliation{Instituto de Cosmologia, Relatividade e Astrof\'{\i}sica,
% Centro Brasileiro  de Pesquisas F\'{\i}sicas.\\
% Rua Dr. Xavier Sigaud 150, Cep 22290-180, Urca, Rio de Janeiro, RJ,
% Brasil.}
\begin{abstract}
In this letter we give the verification of the expressions given in Sect. 4 from
the Alard 2007. We point out, again, that we must include the Einstein Radius adequately
in Eqs. (30) and (31) from the paper mentioned above.
\end{abstract}
\date{\today}
\maketitle
We keep the notation of the Alard Notes (Filling in Details of Alard 2007 MNRAS Paper).
Writing Eq.~(20) in cartesian coordinates ( the unit polar vectors), considering $\vec{y}=(y_1,y_2)$ at the source plane
and $\vec{r}=(r,\te)$ at the lens plane. Don�t forget that as we have not considered the position of the source, 
we have that $\bar{f}_i=f_i$ and therefore
\begin{eqnarray}
y_1 &=& [\kt x - f_1]\cos{\te}+\dfrac{1}{\re}\dfrac{d f_0}{d\te}\sin{\te} \label{y_1}\\
y_2 &=& [\kt x - f_1]\sin{\te}-\dfrac{1}{\re}\dfrac{d f_0}{d\te}\sin{\te} \label{y_2},
\end{eqnarray}
and the Jacobian of the transformation is given by
\begin{equation}
J=\dfrac{\pa y_1}{\pa r}\dfrac{\pa y_2}{\pa \te}-\dfrac{\pa y_1}{\pa \te}\dfrac{\pa y_2}{\pa r}.
\label{jacob}
\end{equation}

The critical lines are defined at the points where $J=0$. Now, is easily to verify that
\begin{eqnarray}
\dfrac{\pa y_1}{\pa r}&=& \dfrac{\kt}{\epsilon} \cos{\te}\label{dy1dx}   \\
\dfrac{\pa y_2}{\pa r}&=& \dfrac{\kt}{\epsilon} \sin{\te}\\
\dfrac{\pa y_1}{\pa \te}&=& \mf(\te)\sin{\te}+\mg(\te)\cos{\te}\\
\dfrac{\pa y_2}{\pa \te}&=& -\mf(\te)\cos{\te}+\mg(\te)\sin{\te}\label{dy2dte}
\end{eqnarray}
here, we have used Eq. (8) of Alard�s Notes , $\pa /\pa r=(1/\epsilon)(\pa /\pa x)$ and the functions $\mf$ and $\mg$ are
\begin{equation}
\mf(\te)=\dfrac{1}{\re}\dfrac{d^2f_0}{d\te^2}-(\kt x -f_1) \quad \textrm{and} \quad %
\mg(\te)=\dfrac{1}{\re}\dfrac{df_0}{d\te}-\dfrac{df_1}{d\te}
\end{equation}

Substituting the Eqs.~(\ref{dy1dx}-\ref{dy2dte}) in Eq.~(\ref{jacob}) is straightforward to verify that $J=-\mf(\te)=0$ and therefore
\begin{equation}
x=\dfrac{1}{\kt}\left[f_1+\dfrac{1}{\re}\dfrac{d^2f_0}{d\te^2}\right] \label{xte}.
\end{equation}
This is Eq. (30) of Alard�s Paper. Now, substituting this last equation in Eqs.~(\ref{y_1})
and (\ref{y_2}) is very easy to get to

\begin{eqnarray*}
y_1 &=& \dfrac{1}{\re}\dfrac{d^2f_0}{d\te^2}\cos{\te}+\dfrac{1}{\re}\dfrac{df_0}{d\te}\sin{\te}\\
y_2 &=& \dfrac{1}{\re}\dfrac{d^2f_0}{d\te^2}\sin{\te}-\dfrac{1}{\re}\dfrac{df_0}{d\te}\cos{\te}\\
\end{eqnarray*}
These last equations are the Eqs.~(31) of Alard 2007


\end{document} 
