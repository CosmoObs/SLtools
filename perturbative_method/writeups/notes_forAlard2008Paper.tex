\documentclass[a4paper,twoside]{article}
\usepackage{latexsym}
% \usepackage[pdftex]{color,graphicx}
\usepackage{latexsym}
\usepackage{subfigure}
\usepackage{amsmath}
\usepackage{ccaption}
\usepackage{natbib}
\usepackage{amssymb}
% \usepackage[brazilian]{babel}
\usepackage[latin1]{inputenc}
\def \re {r_{\mathrm{E}}}
\def \al {\alpha}
\def \te {\theta}
\def \tep {\theta_p}
\def \prtl {\partial}

\begin{document}
\title{Filling in Details of Alard 2008 MNRAS Paper}
%Field perturbation induced by substructures: \\ Verification of Eq.~9 from Alar% d 2008}
%
\author{Collective Brazil DES-SL group}
% email{hdumetm@cbpf.br}
% \affiliation{ICRA-CBPF}
% \author{Mart\'in Makler}
% \email{martin@cbpf.br}
% \affiliation{Instituto de Cosmologia, Relatividade e Astrof\'{\i}sica,
% Centro Brasileiro  de Pesquisas F\'{\i}sicas.\\
% Rua Dr. Xavier Sigaud 150, Cep 22290-180, Urca, Rio de Janeiro, RJ,
% Brasil.}

%\begin{abstract}
%In this letter I shall verify the Eq. (9) from Alard 2008 (Sec. 2.1). As we will
%see at the end of this letter, the expression given in Eq. (9) must
%include the expression for the Angle Deflection.
%\end{abstract}

\date{\today}
\maketitle

\section{Intro}
\subsection{The Pert Approach in SL}

We derived these eq's carefully in our Alard 2007 writeup

\section{Pert Description of the Effects of Substructure on Arcs}

\section{Field Perts Induced by Substructures}

Let us take a perturbation to the central potential with the following properties:

\begin{equation}
\psi(r)=m_p r^\prime
\end{equation}
where $m_p$ is the mass contained within the Einstein radius. From Figure 3 of Alard 2008 (which is one of the worse hand-drawn figures seen in recent times in a published paper -- note how badly the arrowheads are situated), we may express
\begin{eqnarray}
\vec{r}_p&=&r_p\cos{\tep}\hat{\imath} + r_p\sin{\tep}\hat{\jmath}\\
\vec{r}&=&r\cos{\te} \; \hat{\imath} + r\sin{\te}\; \hat{\jmath}\\
\vec{r}^\prime&=&\vec{r}_p-\vec{r}
\end{eqnarray}
where $r$ is the distance to the unit circle (unlabeled by Alard), and $r'$ is the distance from the center of the perturber to the unit circle (what Alard denotes $dr$).
Taking the modulus of $\vec{r}^\prime$, it is straightforward to verify that
the perturbed field is given by
\begin{equation}
\psi(r)=m_p\sqrt{r^2-2r r_p\cos{(\te-\tep)}+r^2_p}.
\end{equation} 

Now, we apply the definitions of the perturbed functions $f_0$ and $f_1$. From Alard07 eq.7:
\begin{equation}
f_0=\psi(\re)=m_p \sqrt{\re^2-2\re r_p\cos{(\te-\tep)}+r^2_p}.
\end{equation}
Taking the derivative of this with respect to $\te$, we obtain
\begin{equation}
\label{eq:df0dt}
\frac{1}{\re}\frac{d f_0}{d\te}=\frac{m_p r_p \sin{(\te -\tep)}}{\sqrt{\re^2-2\re r_p\cos{(\te-\tep)}+r^2_p}}.
\end{equation}

Taking the derivative of $\psi(r)$ with respect to $r$ we have
\begin{equation*}
\frac{\prtl  \psi(r)}{\prtl r}=\frac{m_p[r-r_p\cos{(\te-\tep)}]}{\sqrt{r^2-2r r_p\cos{(\te-\tep)}+r^2_p}},
\end{equation*}
and finally, evaluating this expression at the Einstein Radius, we have again using the definition in Alard07 eq.7:
\begin{equation}
\label{eq:f1}
f_1=\frac{m_p[\re-r_p\cos{(\te-\tep)}]}{\sqrt{\re^2-2\re r_p\cos{(\te-\tep)}+r^2_p}}.
\end{equation}

The expressions in eq.(\ref{eq:df0dt}) and eq.(\ref{eq:f1}) yield Alard 2008 eq.9, generalized for the expression for the Einstein Radius not being set to unity.


\end{document}
