\chapter{Introduction}

\section{Remarks on Notation}

{\bf Source Plane Coordinates}
\begin{itemize}
\item{ $ y = (y_1, y_2) $ are Cartesian coordinates of a point in the source plane}
\item{   $r_s $ is the radial coordinate of a point in the source plane}
\item{   $r_0 $ is the radial coordinate to the center of a circular source in the source plane}

\item{   $\vec{r}_0 =(\yone,\ytwo)$ is the vector to the center of a circular source in the source plane}

\item{ $R_0 $ is the vector distance from the center of circular
  source to its edge in the source plane}

\item{ $\eta_s$ is the ellipticity of an object in the source plane}

\end{itemize}

{\bf Lens Plane Coordinates}

\begin{itemize}
\item{ $ x = (x_1, x_2) $ are Cartesian coordinates of a point in the lens plane, relative to the Einstein radius}
\item{ $  (r, \theta) $ are polar coordinates of a point in the lens plane}
\item{ $\re$ is the Einstein Radius of the critical curve in the lens plane}
\end{itemize}



\section{Summary of primary equations}

From \secref{sec:BasicIdeas}, we have the division of the potential into
axisymmetric and non-axisymmetric pieces:

eq.~(\ref{eq:potsplit}): $\phi  =  \phi_0 + \eps \psi$.

Which we expand into

\eqref{eq:tse2}): $\phi  = \sum_{n=0}^\infty \left[ C_n + \eps f_n  \right] (r-\re)^n$

with definitions

\eqref{eq:Cndef}): $C_n \equiv {1 \over n!} \left. {d^n \phi_0 \over dr^n }\right|_{\re}$

\eqref{eq:fndef}): $f_n(\theta) \equiv  {1 \over n!} \left. {d^n \psi(\theta) \over dr^n }\right|_{\re}$

And this leads to the equation for the mapping of the lens to source plane in terms of the $f_{0,1}$ functions:

\eqref{eq:rsexpanded}):  $y = \left[ \kt x - f_1 \right] \hat{r} - {1 \over \re} {\prtl f_0 \over \prtl \te}  \hat{\theta}$

with  $\kt \equiv 1-2 C_2$
\\
\\
\\
{\bf Lens Curves}
Now we move to the ``lens curves'' -- the equations for the critical curves and caustics.

First, for the tangential critical curves:

\eqref{eq:critcurves}): $x=\frac{1}{\kt}\left[f_1+\frac{1}{\re}\frac{d^2f_0}{d\te^2}\right] \label{xte}$

And for the caustics:

\eqref{eq:xcaustic}): $y_{1_\mathrm{caust}} = \frac{1}{\re}\frac{d^2f_0}{d\te^2}\cos{\te}+\frac{1}{\re}\frac{df_0}{d\te}\sin{\te}$

\eqref{eq:ycaustic}): $y_{2_\mathrm{caust}} = \frac{1}{\re}\frac{d^2f_0}{d\te^2}\sin{\te}-\frac{1}{\re}\frac{df_0}{d\te}\cos{\te}$
\\
\\
\\
{\bf Arcs}

For a circular extended source, the radial arc position in the lens plane is

\eqref{eq:xsoln}):  $x = \frac{1}{\kappa_2}\left[ \overline{f}_{1}(\theta) \pm \sqrt{R_0^2 - \left( \frac{1}{\re}\frac{\partial \overline{f}_0(\theta)}{\partial \theta} \right)^2} \right] $

With $\overline{f}_i(\theta) \equiv f_i(\theta) + (x_0 \cos \te + y_0 \sin \te)\re^{1-i}, \;\; i=0,1 $

And
\eqref{eq:parametricarcs}): $\vec{r}= \left[(\re +x)\cos{\te},(\re+x)\sin{\te}\right], \quad 0 \leq \te < 2\pi$

are the Cartesian coordinates for this in parametric form.
\\
\\
For an elliptical extended source where the source is oriented at an angle $\theta_0$ \wrt\ the semi-major axis of the central potential, the radial arc position in the lens plane is

\eqref{eq:ellipsource}): $x = \frac{1}{\kappa_2} \left\{\bar{f}_{1}(\theta) +%
\frac{\eta_s\sin(2\tilde{\theta})}{S}\left(\frac{1}{\re}\dfrac{\partial%
\bar{f}_0(\theta)}{\partial \theta}\right) \pm \frac{1}{S}\sqrt{SR_{0}^2 -%
(1-\eta_s^2)\left[ \frac{1}{\re}\frac{\partial \bar{f}_0(\theta)}{\partial
\theta}\right]^2}  \right\}$

with $ S \equiv 1-\eta_s\cos{(2\tilde{\te})}$, and $\tilde{\te} = \theta - \theta_0$

And again
\eqref{eq:parametricarcs}): $\vec{r}= \left[(\re +x)\cos{\te},(\re+x)\sin{\te}\right], \quad 0 \leq \te < 2\pi$

are the Cartesian coordinates for this in parametric form.


{\bf }
