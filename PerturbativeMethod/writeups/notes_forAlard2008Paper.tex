\documentclass[a4paper,twoside]{article}
\usepackage{latexsym}
% \usepackage[pdftex]{color,graphicx}
\usepackage{latexsym}
\usepackage{subfigure}
\usepackage{amsmath}
\usepackage{ccaption}
\usepackage{natbib}
\usepackage{amssymb}
% \usepackage[brazilian]{babel}
\usepackage[latin1]{inputenc}
\def \re {r_{\mathrm{E}}}
\def \al {\alpha}
\def \te {\theta}
\def \tep {\theta_p}
\def \prtl {\partial}

\begin{document}
\title{Field perturbation induced by substructures: \\ Verification of Eq.~9 from Alard 2008}
%
\author{Habib S. Dumet-Montoya}
% email{hdumetm@cbpf.br}
% \affiliation{ICRA-CBPF}
% \author{Mart\'in Makler}
% \email{martin@cbpf.br}
% \affiliation{Instituto de Cosmologia, Relatividade e Astrof\'{\i}sica,
% Centro Brasileiro  de Pesquisas F\'{\i}sicas.\\
% Rua Dr. Xavier Sigaud 150, Cep 22290-180, Urca, Rio de Janeiro, RJ,
% Brasil.}
\begin{abstract}
In this letter I shall verify the Eq. (9) from Alard 2008 (Sec. 2.1). As we will
see at the end of this letter, the expression given in Eq. (9) must
include the expression for the Angle Deflection.
\end{abstract}

\date{\today}
\maketitle
We can consider the perturbation as
\begin{equation}
\psi(r)=m_p r^\prime
\end{equation}
where $m_p$ is the mass contained within the Einstein radius. From Figure 3 of Alard 2008, we may express
\begin{eqnarray}
\vec{r}_p&=&r_p\cos{\tep}\hat{\imath} + r_p\sin{\tep}\hat{\jmath}\\
\vec{r}&=&r\cos{\te}\hat{\imath} + r\sin{\te}\hat{\jmath}\\
r^\prime&=&\vec{r}_p-\vec{r}
\end{eqnarray}
Taking the modulus of $r^\prime$, it is straightforward to verify that
the perturbed field is given by
\begin{equation}
\psi(r)=m_p\sqrt{r^2-2r r_p\cos{(\te-\tep)}+r^2_p}.
\end{equation} 
Now, we apply the definitions of the perturbed functions $f_0$ and $f_1$. Then,
\begin{equation}
f_0=\psi(\re)=m_p \sqrt{\re^2-2\re r_p\cos{(\te-\tep)}+r^2_p}.
\end{equation}
Taking the derivative of this with respect to $\te$, we obtain
\begin{equation}
\frac{1}{\re}\frac{d f_0}{d\te}=\frac{m_p r_p \sin{(\te -\tep)}}{\sqrt{\re^2-2\re r_p\cos{(\te-\tep)}+r^2_p}}.
\end{equation}

Taking the derivative of $\psi(r)$ with respect to $r$ we have
\begin{equation*}
\frac{\prtl  \psi(r)}{\prtl r}=\frac{m_p[r-r_p\cos{(\te-\tep)}]}{\sqrt{r^2-2r r_p\cos{(\te-\tep)}+r^2_p}},
\end{equation*}
and finally, evaluating this expression at the Einstein Radius, we have
\begin{equation}
f_1=\frac{m_p[\re-r_p\cos{(\te-\tep)}]}{\sqrt{\re^2-2\re r_p\cos{(\te-\tep)}+r^2_p}}.
\end{equation}

Conclusion: The expression in Eq. (9) of Alard 2008, must include the expression for the Einstein Radius.


\end{document}
