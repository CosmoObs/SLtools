\documentclass[a4paper,twoside,prd]{revtex4}
\usepackage{latexsym}
\usepackage[pdftex]{color,graphicx}
\usepackage{latexsym}
\usepackage{subfigure}
\usepackage{amsmath}
\usepackage{ccaption}
\usepackage{natbib}
\usepackage{amssymb}
\usepackage[brazilian]{babel}
\usepackage[latin1]{inputenc}
\def \re {r_{\mathrm{E}}}
\def \al {\alpha}
\def \te {\theta}
\def \tep {\theta_p}

\begin{document}
\title{Fiedl perturbation induced by substructures: \\ Verification of the Eq.~9 from Alard�s 2008}
%
\author{Habib S. Dumet-Montoya}
% email{hdumetm@cbpf.br}
\affiliation{ICRA-CBPF}
% \author{Mart\'in Makler}
% \email{martin@cbpf.br}
% \affiliation{Instituto de Cosmologia, Relatividade e Astrof\'{\i}sica,
% Centro Brasileiro  de Pesquisas F\'{\i}sicas.\\
% Rua Dr. Xavier Sigaud 150, Cep 22290-180, Urca, Rio de Janeiro, RJ,
% Brasil.}
\begin{abstract}
In this letter I shall verify the Eq. (9) from the Alard�s Paper 2008. As we will se at the end of this letter, the
expression given in Eq. (9) must be include the expression for the Angle Deflection.
\end{abstract}

\date{\today}
\maketitle
We can considerer the perturbation as
\begin{equation}
\psi(r)=m_p r^\prime
\end{equation}
where $m_p$ is the mass contained at the Einstein circle. From the figure (3) from Alard�s paper 2008, we can express
\begin{eqnarray}
\vec{r}_p&=&r_p\cos{\tep}\hat{\imath} + r_p\sin{\tep}\hat{\jmath}\\
\vec{r}&=&r\cos{\te}\hat{\imath} + r\sin{\te}\hat{\jmath}\\
r^\prime&=&\vec{r}_p-\vec{r}
\end{eqnarray}
Taking the modulus of $r^\prime$, is straightforward to verify that the perturbed field is 
given by
\begin{equation}
\psi(r)=m_p\sqrt{r^2-2r r_p\cos{(\te-\tep)}+r^2_p}.
\end{equation} 
Now, we will apply the definition of the perturbed functions $f_0$  (and its related quantity) and $f_1$. Then,
\begin{equation}
f_0=\psi(\re)=m_p \sqrt{\re^2-2\re r_p\cos{(\te-\tep)}+r^2_p}.
\end{equation}
Making the derivative in respect to $\te$, we have
\begin{equation}
\dfrac{1}{\re}\dfrac{d f_0}{d\te}=\dfrac{m_p r_p \sin{(\te -\tep)}}{\sqrt{\re^2-2\re r_p\cos{(\te-\tep)}+r^2_p}}.
\end{equation}

Taking the derivative of the $\psi(r)$ respect to $r$ we have
\begin{equation*}
\dfrac{\psi(r)}{dr}=\dfrac{m_p[r-r_p\cos{(\te-\tep)}]}{\sqrt{r^2-2r r_p\cos{(\te-\tep)}+r^2_p}},
\end{equation*}
and finally, evaluation this expression at Einstein Radius, we have
\begin{equation}
f_1=\dfrac{m_p[\re-r_p\cos{(\te-\tep)}]}{\sqrt{\re^2-2\re r_p\cos{(\te-\tep)}+r^2_p}}.
\end{equation}

Conclussion: The expression in the Eq. (9) of Alard�s Paper 2008, must be include the expression for the Einstein Radius.


\end{document}
